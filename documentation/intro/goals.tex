\headsection{Project presentation}

\subsection{Motivation}

This project begun the previous semester during the Processor Design course.
I decided that the project for said course could be the initial steps for my
master thesis, and after discussing it with the professor we agreed on a topic
that could be developed within the expected course but that might be further 
expanded as my thesis. 

\medbreak
I was conflicted weather to do my thesis about hardware architecture or about
compilers. Taking into account the discussed before I decided to mix both. I 
have always enjoyed the hardware design courses during my degree and later in
the master, but I also felt a great deal of interest in compilers as they are 
the \textit{interpreters} of human intentions to the computers. Therefore, 
being able to work in both fields for my project felt like the most appropriate 
direction. 

\medbreak
My first take on the project was to add predication to an existing processor
and then add the compiler support for this feature. However, after consulting
the RISC-V specification I realized that doing so would not add any significant
improvement \cite{RVSpec}. On the other hand, I was aware of the increasing 
interest in implementing efficient machine learning in space processors, and 
decided to turn on this area but keep my original idea of predication as an 
additional feature. 

\medbreak
With the main topic decided, the selection of the base processor was a natural 
given it had to be an space processor which was available under a public license
and written in a language I know. With this conditions in mind, I selected the
LEON3 \cite{L3} processor designed by Cobham Gaisler. However, recently they have
released the NOEL-V \cite{NV} which follows the RISC-V standard, because of this,
I decided to work using both processors.

\subsection{Objectives}

As introduced in the previous section the main goal for this project is to add 
additional support for efficient machine learning in the space processors LEON3
and NOEL-V. To do so I will follow two different approaches. The first one is to
add a SIMD module that performs vector operations aimed at improving performance
in artificial intelligence applications. By checking the most common instructions
and characteristics in said applications the hardware will be optimized for an
improved performance. 

\medbreak
The second approach is to turn both processors into static 
superscalars with a dual issue pipeline. Furthermore, the integer pipeline will 
be extended with predication support allowing to execute or not instructions 
depending on a predicate value, thus turning control dependencies into data 
dependencies.  Finally, to properly take advantage of the modifications the 
compiler will be modified to use the new instructions in the SIMD module and the
predication characteristics that will now be available. 
