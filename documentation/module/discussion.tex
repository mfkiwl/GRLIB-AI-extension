\subsection{Design decisions}
\subsubsection{Second stage saturation} \label{satdiscus}
Initially there wasn't supposed to be saturation in the second stage,
the reason being that since in it reduction operations were performed,
there was no need to be concerned on the overflow in the result. However,
this produced inconsistencies regarding the input and output data types,
therefore, I decided to add saturation in the second stage.

\medbreak
However, there were no available bits in the instruction to extend the 
second operand code to add saturation support. Thus, I considered three
different approaches to solve this issue. 

\begin{itemize}
    \item At this point of the project it hadn't been considered to add
        the possibility to operate using an immediate, and the bit that 
        identifies the operand as such was kept in regards of coherence 
        with the rest of the ISA.
        
        \medbreak
        Therefore, I first thought of using the immediate bit as a \textit{saturation}
        bit, that for indicated for both stages weather the operation 
        should use saturation or not. However, this meant a loss of many
        bit combinations as there are operations incompatible with 
        saturation. 

    \item Similarly, taking advantage of the immediate bit I considered to
        increase the size of the second opcode to include the immediate bit 
        as well. But this altered the coherence of the instruction and 
        blocked to use in the future of immediate operands and thus the
        idea was also discarded.

    \item Finally, I opted for the final approach, which is the one that
        was finally implemented, that links the saturation of the second
        stage with the first stage. This means that if the instruction on
        the first stage used saturation so will the second stage.

        \medbreak
        There were some considerations to make before deciding for this 
        last option. It may seem that with this approach if there is no
        instruction in the first stage we can't have saturation in the 
        second. However, if in the first stage we do a saturated addition
        with zero there is no issue. 

        \medbreak
        The only drawback, which happens also in the first approach, is 
        that we cannot have saturation in only one stage. Nonetheless, the
        result of said operation can still be achieved by using two consecutive
        instructions. And we also have to consider that when using saturation
        in a data value all consecutive operations will likely also use 
        saturation.
\end{itemize}
