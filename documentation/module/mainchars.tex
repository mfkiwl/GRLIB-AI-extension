\subsection{Main characteristics}
The Single Instruction Multiple Data (SIMD) module, was initially 
conceived as a SIMD within a register (SWAR), this means that it 
will use the same registers as the integer pipeline but will do 
the operations in smaller subsets. 

\medbreak
This characteristic has the advantage of not requiring an additional 
register file to store the larger registers, which is ideal for a 
module with minimal impact on space and power consumption. Also, is 
important to note that artificial intelligence applications work with 
small values of data *CITATION NEEDED*. Therefore, I decided that each
component of the vector register would be 8 bits long. 

\medbreak
The functions implemented were decided through an analysis of which 
operations are used in artificial intelligence applications, and it was seen
that the dot product, is a pivotal operation in machine learning\cite{MLanalysis}.
This gave the conclusion that it would be beneficial to implement in 
a single instruction the multiplication and addition. 

\medbreak
It can be seen that there are also other situations in which it may 
be interesting to perform two operations consecutively. And in many
cases the second part would be a reduction performed on the result of
the first computation. With this in mind I divided the module in two 
stages. Since there is no need for the module to access memory, as it
works using the same registers as the integer pipeline, the second stage 
of the module can match with the memory access stage with no drawback.

\medbreak
The module has also support for saturated instructions, this means that
for those operations the result will not overflow over the data type 
representation. The saturation of the operation is given in the opcode
for the first stage, and in case there is saturation, the same is done
in the second stage. In section \ref{satdiscus} the justification for 
this characteristic is explained in more detail.
 
