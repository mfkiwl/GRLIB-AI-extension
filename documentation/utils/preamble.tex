\usepackage{geometry}
\geometry{left=2.5cm,right=2.5cm,top=2.5cm}
\usepackage[utf8]{inputenc}

%colors del document
\usepackage[immediate]{silence}
\WarningFilter[temp]{latex}{Command} % silence the warning
\usepackage{sectsty}
\DeactivateWarningFilters[temp]
\sectionfont{\color{pantone3005}}
\subsectionfont{\color{gray40}}
\subsubsectionfont{\color{gray60}}

\usepackage{helvet}

\usepackage{longtable}

%evitar trencar paragrafs
\usepackage[defaultlines=10,all]{nowidow}
%colors pel text
\usepackage[table]{xcolor}
\usepackage{enumitem}
\definecolor{pantone3005}{RGB}{0 119 200}
\definecolor{gray40}{gray}{0.4}
\definecolor{gray60}{gray}{0.6}


%fila de la pagina sense columnes amb colors de la fib
\newcommand{\allrow}[2]{
 \rowcolor{gray60}\multicolumn{#1}{|c|}{\textcolor{white}{#2}} \\
 }

\usepackage{pdflscape}
\usepackage{eurosym}
\usepackage{blindtext}

%annexos
\usepackage[toc,page]{appendix}

%format dels index
\usepackage{tocloft}

%color index
\renewcommand{\cfttoctitlefont}{\bf\LARGE\color{pantone3005}}
\renewcommand{\cftloftitlefont}{\bf\LARGE\color{pantone3005}}
\renewcommand{\cftlottitlefont}{\bf\LARGE\color{pantone3005}}
\renewcommand\cftsecfont{\large\color{pantone3005}}

%macro que crea una secció sense numero de pagina en l'Índex
\newcommand{\headsection}[1]{
\addtocontents{toc}{\cftpagenumbersoff{section}}
\section{#1}
\addtocontents{toc}{\cftpagenumberson{section}}
}

\renewcommand{\contentsname}{Table of Contents}


%format url en la bibliografia
\usepackage[hyphens]{url}
\usepackage[hidelinks]{hyperref}
\hypersetup{breaklinks=true}

%indentar primer paràgraf de la secció
\usepackage{indentfirst}

%format de la descripció de figures i taules 
\usepackage[font=footnotesize,labelfont={bf, color = pantone3005},textfont=it]{caption}

%referencies format [n] en el text i ordre d'aparicio
\usepackage[square,numbers]{natbib}
\bibliographystyle{unsrtnat}

%pagina en blanc
\newcommand{\blankpage}{\clearpage\mbox{}\thispagestyle{nonum}\clearpage}
\usepackage{graphicx}
